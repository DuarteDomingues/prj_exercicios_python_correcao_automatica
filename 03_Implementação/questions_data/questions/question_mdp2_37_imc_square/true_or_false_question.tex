\documentclass[12pt,varwidth=16cm,border=17pt]{standalone}

\usepackage[T1]{fontenc}
\usepackage[utf8]{inputenc}
\usepackage[portuguese]{babel}


\usepackage{listings}

\usepackage[%
    left=1.02in,%
    right=0.50in,%
    top=1.0in,%
    bottom=1.0in,%
]{geometry}%

\usepackage{listings}
\usepackage{xcolor}

\definecolor{codegreen}{rgb}{0,0.6,0}
\definecolor{codegray}{rgb}{0.5,0.5,0.5}
\definecolor{codepurple}{rgb}{0.58,0,0.82}
\definecolor{backcolour}{rgb}{0.96,0.96,0.96}

\lstdefinestyle{mystyle}{
    backgroundcolor=\color{backcolour}, 
    commentstyle=\color{codegreen},
    keywordstyle=\color{magenta},
    numberstyle=\scriptsize\color{codegray},
    stringstyle=\color{codepurple},
    basicstyle=\ttfamily\normalsize,
    breakatwhitespace=false,         
    breaklines=true,                 
    captionpos=b,                    
    keepspaces=true,                 
    numbers=left,                    
    numbersep=7pt,
    showspaces=false,                
    showstringspaces=false,
    showtabs=false,                  
    tabsize=2,
    frame = single,
}

\lstset{style=mystyle}

\begin{document}

Em Pyhton 3, escreva e adicione ao seguinte programa a função \textbf{imc}, que retorna o valor do índice de massa corporal, consoante um argumento \verb+altura+ e um argumento \verb+massa+.
\begin{itemize}
    
  \item Em Python 3, o operador de exponenciação (ou potência) é \verb+**+. Permite obter $potencia=base^{expoente}$ usando potencia = base ** expoente.
  
  
  
  \item  O índice de massa corporal, imc, de uma pessoa, é dado pela fórmula

  \emph{imc=massa÷$altura^2$}. Onde \emph{massa} é o peso da pessoa, em Kg, e a \emph{altura} é a altura da pessoa, em m (metros);
  \item A função \textbf{imc} irá ter como primeiro argumento \verb+altura+ e como segundo argumento \verb+massa+. A função \textbf{imc} retorna o imc correpondente ao peso e à altura. Use o operador de potência para calcular $altura^2$;
  \item O valor retornado para o imc será arredondado para a segunda casa décimal;
  \item A função \verb+seed+, é utilizada para inicializar um gerador de número aleatórios;
  \item A função uniform, da biblioteca random, é utilizada para gerar valores do tipo float aleatórios num intervalo especifico.
  
  

    
\end{itemize}




Considere a execução do programa Python 3, que se segue. 

%[language=Python, caption=Python example]%
\begin{lstlisting}[language=Python]
import random

random.seed(400)

lista_imc = []
for i in range (1000):
    altura = round(random.uniform(1,2),2)
    massa  = round(random.uniform(40,80),2)
    lista_imc.append(imc(altura,massa))

print(lista_imc[420])
print(type(6**5))
\end{lstlisting}

Indique se é verdadeiro ou falso.
\end{document}