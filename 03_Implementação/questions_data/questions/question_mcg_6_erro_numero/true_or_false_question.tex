\documentclass[12pt,varwidth=16cm,border=17pt]{standalone}

\usepackage[T1]{fontenc}
\usepackage[utf8]{inputenc}
\usepackage[portuguese]{babel}


\usepackage{listings}

\usepackage[%
    left=1.02in,%
    right=0.50in,%
    top=1.0in,%
    bottom=1.0in,%
]{geometry}%

\usepackage{listings}
\usepackage{xcolor}

\definecolor{codegreen}{rgb}{0,0.6,0}
\definecolor{codegray}{rgb}{0.5,0.5,0.5}
\definecolor{codepurple}{rgb}{0.58,0,0.82}
\definecolor{backcolour}{rgb}{0.96,0.96,0.96}

\lstdefinestyle{mystyle}{
    backgroundcolor=\color{backcolour}, 
    commentstyle=\color{codegreen},
    keywordstyle=\color{magenta},
    numberstyle=\scriptsize\color{codegray},
    stringstyle=\color{codepurple},
    basicstyle=\ttfamily\normalsize,
    breakatwhitespace=false,         
    breaklines=true,                 
    captionpos=b,                    
    keepspaces=true,                 
    numbers=left,                    
    numbersep=7pt,
    showspaces=false,                
    showstringspaces=false,
    showtabs=false,                  
    tabsize=2,
    frame = single,
}

\lstset{style=mystyle}

\begin{document}

Em Pyhton 3, adicione ao seguinte código o construtor da classe \verb+Numero+ e o método \verb+ordenar_lista_decresente+:

\begin{itemize}

  \item Para além de self, o construtor da classe \verb+Numero+, tem o argumento \verb+valor+;
  \item Os objetos do tipo \verb+Numero+ têm o atributo \verb+valor+;
  \item O atributo \verb+valor+ é inicializado, no construtor, com o valor
  do argumento \verb+valor+;
  \item os objetos do tipo \verb+Numero+ têm um método \verb+ordenar_lista_decresente.+ O
    método \verb+ordenar_lista_decresente+ recebe um argumento chamado \verb+lista+ do tipo
  \verb+list+ que ordena uma lista de forma decrescente e retorna-a.
    
  \item a função \verb+seed+, é utilizada para inicializar um gerador de número aleatórios.
    
\end{itemize}

Considere a execução do seguinte código Python 3.


%[language=Python, caption=Python example]%
\begin{lstlisting}[language=Python]
import random
random.seed(50)

class Numero:

    def generate_random_lists(self, lists_length):
        lista = []
        for i in range (self.valor):
            lista_i = (random.sample(range(1, 40), lists_length))
            lista_ordenada = self.ordenar_lista_decresente(lista_i)
            lista.append(lista_ordenada)
        
        return lista


a = Numero(70)
print (a.generate_random_lists(5)[6][2])
\end{lstlisting}

Indique se é verdadeiro ou falso.
\end{document}