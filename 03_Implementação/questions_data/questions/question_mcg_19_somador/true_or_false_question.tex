\documentclass[12pt,varwidth=16cm,border=17pt]{standalone}

\usepackage[T1]{fontenc}
\usepackage[utf8]{inputenc}
\usepackage[portuguese]{babel}


\usepackage{listings}

\usepackage[%
    left=1.02in,%
    right=0.50in,%
    top=1.0in,%
    bottom=1.0in,%
]{geometry}%

\usepackage{listings}
\usepackage{xcolor}

\definecolor{codegreen}{rgb}{0,0.6,0}
\definecolor{codegray}{rgb}{0.5,0.5,0.5}
\definecolor{codepurple}{rgb}{0.58,0,0.82}
\definecolor{backcolour}{rgb}{0.96,0.96,0.96}

\lstdefinestyle{mystyle}{
    backgroundcolor=\color{backcolour}, 
    commentstyle=\color{codegreen},
    keywordstyle=\color{magenta},
    numberstyle=\scriptsize\color{codegray},
    stringstyle=\color{codepurple},
    basicstyle=\ttfamily\normalsize,
    breakatwhitespace=false,         
    breaklines=true,                 
    captionpos=b,                    
    keepspaces=true,                 
    numbers=left,                    
    numbersep=7pt,
    showspaces=false,                
    showstringspaces=false,
    showtabs=false,                  
    tabsize=2,
    frame = single,
	literate = {á}{{\'a}}1 {é}{{\'e}}1 {í}{{\'i}}1 {ó}{{\'o}}1 {ú}{{\'u}}1
        {Á}{{\'A}}1 {É}{{\'E}}1  {Í}{{\'I}}1 {Ó}{{\'O}}1  {Ú}{{\'U}}1
        {à}{{\`a}}1 {è}{{\`e}}1 {ì}{{\`i}}1 {ò}{{\`o}}1 {ù}{{\`u}}1
        {À}{{\`A}}1 {È}{{\'E}}1 {Ì}{{\`I}}1 {Ò}{{\`O}}1 {Ù}{{\`U}}1
        {â}{{\^a}}1 {ê}{{\^e}}1 {î}{{\^i}}1 {ô}{{\^o}}1 {û}{{\^u}}1
        {Â}{{\^A}}1 {Ê}{{\^E}}1 {Î}{{\^I}}1 {Ô}{{\^O}}1 {Û}{{\^U}}1
        {ç}{{\c c}}1 {Ç}{{\c C}}1 
        {ã}{{\~a}}1 {õ}{{\~o}}1 {Ã}{{\~A}}1 {Õ}{{\~O}}1,
}

\lstset{style=mystyle}

\begin{document}

Em Pyhton 3, complete a classe Somador, no ficheiro \verb+somador.py+:

\begin{itemize}
  \item a classe possui um construtor, e dois métodos. O método \verb+soma_lista+ e o método estatisticas;
  \item o método \verb+soma_lista+ deve ser implementado, onde recebe como argumento: \verb+self+ e uma lista de números designado por \verb+lista+. O retorno do método é o somatório de todos os números dentro da lista;
  \item dentro do método \verb+soma_lista+ deve utilizar os atributos \verb+max+, \verb+min+, \verb+tota+l, \verb+parcelas+ e \verb+listas+. Os dois \verb+underscores+ antes do nome de cada atributo deve ser ignorado, servem somente para atribuir visibilidade privada aos atributos;
  \item o atributo \verb+max+ devolve o maior número dentro de todas as listas utilizadas no método \verb+soma_lista+, o atributo \verb+min+ devolve o menor número dentro das listas, o atributo \verb+total+ devolve a soma de todos os números das listas utilizadas, o atributo \verb+parcelas+ corresponde ao número total de índices das listas utilizadas e o atributo \verb+listas+ ao número de vezes que foi chamado o método \verb+soma_lista+;
\end{itemize}

Considere a execução do programa Python 3, que se segue. 

%[language=Python, caption=Python example]%
\begin{lstlisting}[language=Python]
import random

random.seed(100)

class Somador:

	def __init__(self):

		self.__max         = 0
		self.__min         = 0
		self.__total       = 0
		self.__parcelas    = 0
		self.__listas      = 0

	def estatisticas(self):

		print("numero de listas somadas = "+str(self.__listas))
		print("numero parcelas somadas = "+str(self.__parcelas))
		print("total somado = "+str(self.__total))
		print("parcela minima = "+str(self.__min))
		print("parcela maxima = "+str(self.__max))
\end{lstlisting}

Indique se é verdadeiro ou falso.
\end{document}