\documentclass[12pt,varwidth=16cm,border=17pt]{standalone}

\usepackage[T1]{fontenc}
\usepackage[utf8]{inputenc}
\usepackage[portuguese]{babel}


\usepackage{listings}

\usepackage[%
    left=1.02in,%
    right=0.50in,%
    top=1.0in,%
    bottom=1.0in,%
]{geometry}%

\usepackage{listings}
\usepackage{xcolor}

\definecolor{codegreen}{rgb}{0,0.6,0}
\definecolor{codegray}{rgb}{0.5,0.5,0.5}
\definecolor{codepurple}{rgb}{0.58,0,0.82}
\definecolor{backcolour}{rgb}{0.96,0.96,0.96}

\lstdefinestyle{mystyle}{
    backgroundcolor=\color{backcolour}, 
    commentstyle=\color{codegreen},
    keywordstyle=\color{magenta},
    numberstyle=\scriptsize\color{codegray},
    stringstyle=\color{codepurple},
    basicstyle=\ttfamily\normalsize,
    breakatwhitespace=false,         
    breaklines=true,                 
    captionpos=b,                    
    keepspaces=true,                 
    numbers=left,                    
    numbersep=7pt,
    showspaces=false,                
    showstringspaces=false,
    showtabs=false,                  
    tabsize=2,
    frame = single,
}

\lstset{style=mystyle}

\begin{document}


Sejam C, a temperatura em graus Celsius, e \emph{F}, a temperatura em graus Fahrenheit. A conversão de graus Celsius para graus Fahrenheit é dada pela fórmula F= 1,8C+ 32. A conversão de graus Fahrenheit para graus Celsius é dada pela fórmula C= \verb+(F-32)/1,8+.
Em Pyhton 3, adicione ao seguinte programa e escrevas as funções \textbf{c2f} e \textbf{f2c} 


\begin{itemize}
    
  \item A função \verb+c2f+ tem um argumento, o valor da temperatura em graus Celsius, e retorna o valor da temperatura em graus Fahrenheit.
  
  \item  A função \verb+f2c+ tem um argumento, o valor da temperatura em graus Fahrenheit, e retorna o valor da temperatura em graus Celsius.

  \item A função \verb+seed+, é utilizada para inicializar um gerador de número aleatórios;
  
  \item A função uniform, da biblioteca random, é utilizada para gerar valores do tipo float aleatórios num intervalo especifico.

    
\end{itemize}




Considere a execução do programa Python 3, que se segue. 

%[language=Python, caption=Python example]%
\begin{lstlisting}[language=Python]
import random
random.seed(99541)

lista_f2c = []
for i in range (1650):

    lista_f2c.append(f2c(round(random.uniform(30,100),2)))

lista_c2f = []
for i in range (1666):

    lista_c2f.append(c2f(round(random.uniform(1,38),2)))

print(lista_c2f[1029])
print(lista_f2c[1448])
\end{lstlisting}

Indique se é verdadeiro ou falso.
\end{document}