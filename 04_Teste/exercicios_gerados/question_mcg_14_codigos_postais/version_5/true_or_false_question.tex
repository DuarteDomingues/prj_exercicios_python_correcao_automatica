\documentclass[12pt,varwidth=16cm,border=17pt]{standalone}

\usepackage[T1]{fontenc}
\usepackage[utf8]{inputenc}
\usepackage[portuguese]{babel}


\usepackage{listings}

\usepackage[%
    left=1.02in,%
    right=0.50in,%
    top=1.0in,%
    bottom=1.0in,%
]{geometry}%

\usepackage{listings}
\usepackage{xcolor}

\definecolor{codegreen}{rgb}{0,0.6,0}
\definecolor{codegray}{rgb}{0.5,0.5,0.5}
\definecolor{codepurple}{rgb}{0.58,0,0.82}
\definecolor{backcolour}{rgb}{0.96,0.96,0.96}

\lstdefinestyle{mystyle}{
    backgroundcolor=\color{backcolour}, 
    commentstyle=\color{codegreen},
    keywordstyle=\color{magenta},
    numberstyle=\scriptsize\color{codegray},
    stringstyle=\color{codepurple},
    basicstyle=\ttfamily\normalsize,
    breakatwhitespace=false,         
    breaklines=true,                 
    captionpos=b,                    
    keepspaces=true,                 
    numbers=left,                    
    numbersep=7pt,
    showspaces=false,                
    showstringspaces=false,
    showtabs=false,                  
    tabsize=2,
    frame = single,
}

\lstset{style=mystyle}

\begin{document}

Em Pyhton 3, escreva a classe \verb+GestorCodigoPostais+, no ficheiro \verb+gestor_codigos_postais.py+:

\begin{itemize}

  \item Para além de self, o construtor da classe \verb+GestorCodigoPostais+, tem o argumento \verb+codigos_postais+;
  \item Os objetos do tipo \verb+GestorCodigoPostais+ têm o atributo \verb+codigos_postais+, que é uma \verb+lista+ de objetos do tipo \verb+CodigoPostal+ ;
  
  \item O atributo \verb+codigos_postais+ é inicializado, no construtor, com o valor do argumento \verb+codigos_postais+;
  
  \item Os objetos do tipo \verb+codigos_postais+ têm os métodos \verb+validar_codigos_postais+ e \verb+obter_codigos_postais_por_localidade+;
  
  \item O método \verb+validar_codigos_postais+ valida o atributo \verb+codigos_postais+, de forma ao atributo \verb+codigos_postais+ conter apenas códigos postais válidos;
  \item Para um código postal ser válido o primeiro número tem que ter quatro digitos, o segundo número tem que ter três digitos e a localidade tem que estar apenas em letras capitais;
  
  \item O método \verb+obter_codigos_postais_por_localidade+, recebe o argumento \verb+localidade+, o método retorna uma lista de objetos do tipo CodigoPostal do atributo \verb+codigos_postais+ cujo a localidade é igual ao argumento \verb+localidade+ recebido;


  \item A função \verb+seed+, é utilizada para inicializar um gerador de número aleatórios;
    
\end{itemize}

%[language=Python, caption=Python example]%
\begin{lstlisting}[language=Python]
import random
import string
from gestor_codigos_postais import GestorCodigoPostais

random.seed(69986)

class CodigoPostal:

    def __init__(self, digitos4, digitos3, localidade):

        self.digitos4             = digitos4
        self.digitos3             = digitos3
        self.localidade           = localidade
        self.separador_digitos    = '-'
        self.separador_localidade = ' '
    
    def print_codigo_postal(self):

        print(f"{self.digitos4}{self.separador_digitos}{self.digitos3}{self.separador_localidade}{self.localidade}")

        

list_localidades = ["LOURES","MAFRA","OEIRAS","CASCAIS","lisboa","ESPINHO","MAIA","Amarante","valongo","OVAR","Pombal","Batalha"]

lista_codigos_postais = []
for i in range (1000):

    numero_4_digitos =  random.randint(1,12836)
    numero_3_digitos =  random.randint(1,1133)
    localidade       =  random.choice(list_localidades)
    codigo_postal_i  =  CodigoPostal(numero_4_digitos,numero_3_digitos,localidade)

    lista_codigos_postais.append(codigo_postal_i)


g = GestorCodigoPostais(lista_codigos_postais)
g.validar_codigos_postais()
print(len(g.codigos_postais))
g.codigos_postais[227].print_codigo_postal()
codigos_postais_localidade = g.obter_codigos_postais_por_localidade("OEIRAS")
print(len(codigos_postais_localidade))
\end{lstlisting}

Indique se as seguintes perguntas são verdadeiras ou falsas.

\end{document}