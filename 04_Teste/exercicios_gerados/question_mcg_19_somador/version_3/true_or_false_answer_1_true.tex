\documentclass[37pt,varwidth=16cm,border=17pt]{standalone}

\usepackage[T1]{fontenc}
\usepackage[utf8]{inputenc}
\usepackage[portuguese]{babel}


\usepackage{listings}

\usepackage[%
    left=1.02in,%
    right=0.50in,%
    top=1.0in,%
    bottom=1.0in,%
]{geometry}%

\usepackage{listings}
\usepackage{xcolor}

\definecolor{codegreen}{rgb}{0,0.6,0}
\definecolor{codegray}{rgb}{0.5,0.5,0.5}
\definecolor{codepurple}{rgb}{0.58,0,0.82}
\definecolor{backcolour}{rgb}{0.96,0.96,0.96}

\lstdefinestyle{mystyle}{
    backgroundcolor=\color{backcolour}, 
    commentstyle=\color{codegreen},
    keywordstyle=\color{magenta},
    numberstyle=\scriptsize\color{codegray},
    stringstyle=\color{codepurple},
    basicstyle=\ttfamily\normalsize,
    breakatwhitespace=false,         
    breaklines=true,                 
    captionpos=b,                    
    keepspaces=true,                 
    numbers=left,                    
    numbersep=7pt,
    showspaces=false,                
    showstringspaces=false,
    showtabs=false,                  
    tabsize=2,
    frame = single,
	literate = {á}{{\'a}}1 {é}{{\'e}}1 {í}{{\'i}}1 {ó}{{\'o}}1 {ú}{{\'u}}1
        {Á}{{\'A}}1 {É}{{\'E}}1  {Í}{{\'I}}1 {Ó}{{\'O}}1  {Ú}{{\'U}}1
        {à}{{\`a}}1 {è}{{\`e}}1 {ì}{{\`i}}1 {ò}{{\`o}}1 {ù}{{\`u}}1
        {À}{{\`A}}1 {È}{{\'E}}1 {Ì}{{\`I}}1 {Ò}{{\`O}}1 {Ù}{{\`U}}1
        {â}{{\^a}}1 {ê}{{\^e}}1 {î}{{\^i}}1 {ô}{{\^o}}1 {û}{{\^u}}1
        {Â}{{\^A}}1 {Ê}{{\^E}}1 {Î}{{\^I}}1 {Ô}{{\^O}}1 {Û}{{\^U}}1
        {ç}{{\c c}}1 {Ç}{{\c C}}1 
        {ã}{{\~a}}1 {õ}{{\~o}}1 {Ã}{{\~A}}1 {Õ}{{\~O}}1,
}

\lstset{style=mystyle}

\begin{document}


Perante a seguinte execução de Python 3. 

%[language=Python, caption=Python example]%
\begin{lstlisting}[language=Python]
somador = Somador()
print(somador.soma_lista([random.randint(14,273), random.randint(24,121), random.randint(24,140)]))
print(somador.soma_lista([-1*random.randint(1345,5031), -2*random.randint(1189, 6392)]))
print(somador.soma_lista([random.randint(14,156), random.randint(37,297)]))
somador.estatisticas()

somador2 = Somador()
print(somador2.soma_lista([random.randint(15,171), random.randint(28,248)]))
print(somador2.soma_lista([-1*random.randint(1401,6538), -2*random.randint(1691, 5503), -3*random.randint(1283,7453)]))
print(somador2.soma_lista([random.randint(25,176), random.randint(36,329), random.randint(7,159), random.randint(5,313)]))
somador2.estatisticas()
\end{lstlisting}



Após implementar o método em falta da classe o output produzido pelas linhas 2, 3, 4 e 5 deve ser:
\newline 
\verb+296+\newline
\verb+-14484+\newline
\verb+152+\newline
número de listas somadas = \verb+3+\newline
número parcelas somadas  = \verb+7+\newline
total somado             = \verb+-14036+\newline
parcela mínima           = \verb+-9618+\newline
parcela máxima           = \verb+127+\newline
\newline
\newline
O output produzido pelas linhas 8, 9, 10 e 11 é:
\newline
\verb+176+\newline
\verb+-20134+\newline
\verb+455+\newline
número de listas somadas = \verb+3+\newline
número parcelas somadas  = \verb+9+\newline
total somado             = \verb+-19503+\newline
parcela mínima           = \verb+-8505+\newline
parcela máxima           = \verb+168+\newline

\end{document}