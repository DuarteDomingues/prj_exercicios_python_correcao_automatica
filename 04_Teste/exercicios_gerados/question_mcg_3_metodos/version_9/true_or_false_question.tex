\documentclass[12pt,varwidth=16cm,border=17pt]{standalone}

\usepackage[T1]{fontenc}
\usepackage[utf8]{inputenc}
\usepackage[portuguese]{babel}


\usepackage{listings}

\usepackage[%
    left=1.02in,%
    right=0.50in,%
    top=1.0in,%
    bottom=1.0in,%
]{geometry}%

\usepackage{listings}
\usepackage{xcolor}

\definecolor{codegreen}{rgb}{0,0.6,0}
\definecolor{codegray}{rgb}{0.5,0.5,0.5}
\definecolor{codepurple}{rgb}{0.58,0,0.82}
\definecolor{backcolour}{rgb}{0.96,0.96,0.96}

\lstdefinestyle{mystyle}{
    backgroundcolor=\color{backcolour}, 
    commentstyle=\color{codegreen},
    keywordstyle=\color{magenta},
    numberstyle=\scriptsize\color{codegray},
    stringstyle=\color{codepurple},
    basicstyle=\ttfamily\normalsize,
    breakatwhitespace=false,         
    breaklines=true,                 
    captionpos=b,                    
    keepspaces=true,                 
    numbers=left,                    
    numbersep=7pt,
    showspaces=false,                
    showstringspaces=false,
    showtabs=false,                  
    tabsize=2,
    frame = single,
}

\lstset{style=mystyle}

\begin{document}

Em Pyhton 3, construa a classe \verb+X+ com os seguintes requesitos:

\begin{itemize}

  \item Para além de self, o construtor da classe \verb+X+, tem o argumento \verb+x+;
  \item Os objetos do tipo \verb+X+ têm o atributo \verb+x+;
  \item O atributo \verb+x+ é inicializado, no construtor, com o valor
	do argumento \verb+x+;
  \item Os objetos do tipo \verb+X+ têm um método \verb+y+. O
    método \verb+y+ possui somente o argumento self e retorna o valor do atributo \verb+x+.
	
  \item Os objetos do tipo \verb+X+ têm um método \verb+z+. O
    método \verb+z+ recebe como argumento o self e um argumento \verb+y+. 
	O método \verb+z+ realiza a afetação do atributo \verb+x+ pelo valor do argumento \verb+y+.
    
 \item Os objetos do tipo \verb+X+ têm um método \verb+obter_contagem_string.+ O
    método além do self, recebe um argumento \verb+lista+ do tipo list, e um argumento \verb+letra+ do tipo string. O método \verb+obter_contagem_string+ retorna o número de ocorrências do argumento letra na lista.
    
 \item A função \verb+seed+, é utilizada para inicializar um gerador de número aleatórios.
    
\end{itemize}

Considere a execução do seguinte código Python 3.






%[language=Python, caption=Python example]%
\begin{lstlisting}[language=Python]
import random
import string

random.seed(71310)

lista = [''.join(random.choice(string.ascii_lowercase) for i in range(2)) for i in range(300)]
x = X('x')
y = X('x')
z = X('x')
print(x.obter_contagem_string(lista,x.y()))
print(x.y())
print(y.y())
print(z.y())
print(type(x.y()))
x.z('s')
y.z('i')
print(x.y())
print(y.y())
print(z.y())
print(type(y))
\end{lstlisting}

Indique se é verdadeiro ou falso.
\end{document}