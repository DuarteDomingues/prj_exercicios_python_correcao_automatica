\documentclass[12pt,varwidth=16cm,border=17pt]{standalone}

\usepackage[T1]{fontenc}
\usepackage[utf8]{inputenc}
\usepackage[portuguese]{babel}


\usepackage{listings}

\usepackage[%
    left=1.02in,%
    right=0.50in,%
    top=1.0in,%
    bottom=1.0in,%
]{geometry}%

\usepackage{listings}
\usepackage{xcolor}

\definecolor{codegreen}{rgb}{0,0.6,0}
\definecolor{codegray}{rgb}{0.5,0.5,0.5}
\definecolor{codepurple}{rgb}{0.58,0,0.82}
\definecolor{backcolour}{rgb}{0.96,0.96,0.96}

\lstdefinestyle{mystyle}{
    backgroundcolor=\color{backcolour}, 
    commentstyle=\color{codegreen},
    keywordstyle=\color{magenta},
    numberstyle=\scriptsize\color{codegray},
    stringstyle=\color{codepurple},
    basicstyle=\ttfamily\normalsize,
    breakatwhitespace=false,         
    breaklines=true,                 
    captionpos=b,                    
    keepspaces=true,                 
    numbers=left,                    
    numbersep=7pt,
    showspaces=false,                
    showstringspaces=false,
    showtabs=false,                  
    tabsize=2,
    frame = single,
}

\lstset{style=mystyle}

\begin{document}

Considere a execução do programa Python 3, que se segue. 

%[language=Python, caption=Python example]%
\begin{lstlisting}[language=Python]
h1 = None
h2 = '811.0'
h3 = [453, 600, 811]
h4 = True
h5 = False
h6 = ['[]']
h7 = 2.0
h8 = 453
h9 = (False, True)
h10 = 'True'
print(type(h1))
print(type(h2))
print(type(h3))
print(type(h4))
print(type(h5))
print(type(h6))
print(type(h7))
print(type(h8))
print(type(h9))
print(type(h10))
\end{lstlisting}

Indique se é verdadeiro ou falso.
\end{document}