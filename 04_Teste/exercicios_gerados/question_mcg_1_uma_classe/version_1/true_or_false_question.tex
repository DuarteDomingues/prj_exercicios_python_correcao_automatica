\documentclass[12pt,varwidth=16cm,border=17pt]{standalone}

\usepackage[T1]{fontenc}
\usepackage[utf8]{inputenc}
\usepackage[portuguese]{babel}


\usepackage{listings}

\usepackage[%
    left=1.02in,%
    right=0.50in,%
    top=1.0in,%
    bottom=1.0in,%
]{geometry}%

\usepackage{listings}
\usepackage{xcolor}

\definecolor{codegreen}{rgb}{0,0.6,0}
\definecolor{codegray}{rgb}{0.5,0.5,0.5}
\definecolor{codepurple}{rgb}{0.58,0,0.82}
\definecolor{backcolour}{rgb}{0.96,0.96,0.96}

\lstdefinestyle{mystyle}{
    backgroundcolor=\color{backcolour}, 
    commentstyle=\color{codegreen},
    keywordstyle=\color{magenta},
    numberstyle=\scriptsize\color{codegray},
    stringstyle=\color{codepurple},
    basicstyle=\ttfamily\normalsize,
    breakatwhitespace=false,         
    breaklines=true,                 
    captionpos=b,                    
    keepspaces=true,                 
    numbers=left,                    
    numbersep=7pt,
    showspaces=false,                
    showstringspaces=false,
    showtabs=false,                  
    tabsize=2,
    frame = single,
}

\lstset{style=mystyle}

\begin{document}


Em Pyhton 3, complete a classe UmaClasse, no ficheiro \verb+uma_class.py+:

\begin{itemize}
  \item para além de self, o construtor da classe UmaClasse, tem o argumento \verb+um_argumento+;
  \item os objetos do tipo UmaClasse têm o atributo \verb+um_atributo+;
  \item o atributo \verb+um_atributo+ é inicializado, no construtor, com o valor
do argumento \verb+um_argumento+;
  \item os objetos do tipo UmaClasse têm um método \verb+um_metodo.+ O
    método \verb+um_metodo+ retorna o atributo \verb+um_atributo+, com tamanho 6 desde o índice \verb+0+ até ao \verb+5+, do objeto self.
\end{itemize}

Considere a execução do seguinte código Python 3.



%[language=Python, caption=Python example]%
\begin{lstlisting}[language=Python]
import string

seed = 72182

def pseudo_random_integer(min_int, max_int):
    global seed
    seed = (16807*seed) % 2147483647
    return int(min_int + (max_int - min_int) * seed / 2147483646)



class UmaClasse():

	def gerar_lista_aleatoria(self):

		strings_aleatorias = []

		for i in range(2165):

			strings_aleatorias.append(self.um_metodo()[pseudo_random_integer(0, 5)])

		return strings_aleatorias


string_completa = string.ascii_lowercase+string.ascii_uppercase
string_aleatoria = ''.join([string_completa[pseudo_random_integer(14,31)] for i in range(2165)])
lista_numeros_aleatorio = [pseudo_random_integer(10000,99999) for i in range(2165)]

objeto1 = UmaClasse(string_aleatoria)
objeto2 = UmaClasse(lista_numeros_aleatorio)

print(objeto1.gerar_lista_aleatoria()[26])
print(objeto2.gerar_lista_aleatoria()[10])

l = objeto2
g = l.um_metodo()
print(g)

l = objeto1
g= l.um_metodo()
print(g)
print(type(l))
print(type(g))
\end{lstlisting}

Indique se é verdadeiro ou falso.
\end{document}